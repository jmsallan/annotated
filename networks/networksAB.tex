\documentclass[12pt]{article}
\usepackage{geometry}   
\geometry{a4paper}
\usepackage[english]{babel}
\usepackage[utf8]{inputenc}
\title{Annotated bibliography on complex networks}
\author{Jose M Sallan}
\date{}
\begin{document}

\maketitle

\section{Network models}

The first attempt to model real-complex networks was the Erdös-R\'enyi (ER) model of random networks. That model does not represent adequately real-world networks. \medskip

As an explanation of real-world network being more clustered than random networks, Watts and Strogatz \cite{Watts1998a} defined the small-world property of complex networks: networks having this property have an average path length similar to random networks, but a much higher clustering coefficient. Later, Latora and colleagues \cite{Latora2001} reformulated the small-world property in terms of global and local efficiency. \medskip

Barab\'asi and colleagues introduced the BA scale-free model of complex networks, where the mechanisms of growth and preferential attachment lead to a degree distribution following a power law \cite{BarabasiA.-L;Albert1999}. Arguably, the behavior of the world-wide-web can be explained with this model \cite{Barabasi2000}. The BA model has a low clustering coefficient, contrarily to random networks, then in \cite{Klemm2002} is defined a KE generative model of highly-clustered scale-free networks. \medskip

Empirical research identified three structural classes of small-world networks: scale-free networks, with a degree distribution with a tail that decays as a power law, broad-scale or truncated scale-free, with a power-law regime followed by a  sharper cutoff and single-scale networks, with a  fast-decaying tail \cite{Amaral2000}. The emergence of the first two networks can be explained by optimization mechanisms, complementary to growth and preferential attachment \cite{Cancho2003}.

\section{Static robustness}

Introduction of the problem of static robustness: \cite{Albert2000}. \medskip

Analysis of static robustness in terms of network efficiency: \cite{Crucitti2003}  for BA and KE graphs, \cite{Crucitti2004b} for BA graphs. Scale-free tend to be robust to errors, but sensitive to attacks. \medskip

In \cite{Paul2005} is obtained the network that optimizes robustness for the sum of critical threshold for random deletion \cite{Cohen2000a} and for deletion of central nodes \cite{Cohen2001a}. \medskip

\section{Cascading failures}

In \cite{Motter2002} is defined a first model of cascading failures. Load of a node is equal to number of shortest paths passing through it. Nodes with load above capacity are removed from the network. Evolution is tracked through size of connected component. \medskip

In the model of \cite{Crucitti2004} nodel load is also equal to number of shortest paths passing through it. They define a mechanism of load redistribution, and track evolution of damage through efficiency. \medskip

\nocite{*}

\bibliographystyle{plain-annote}
\bibliography{bib_networks}
\end{document}