\documentclass[12pt]{article}
\usepackage{geometry}   
\geometry{a4paper, top=2.5cm, bottom=2cm, right=2.5cm, left=2.5cm}
\usepackage[english]{babel}
\usepackage[utf8]{inputenc}
\usepackage{hyperref}
\title{Annotated bibliography on complex networks}
\author{Jose M Sallan}
\date{}
\begin{document}

\maketitle

\section{Network models}

The first attempt to model real-complex networks was the Erdös-R\'enyi (ER) model of random networks. That model does not represent adequately real-world networks. \medskip

As an explanation of real-world network being more clustered than random networks, Watts and Strogatz \cite{Watts1998a} defined the small-world property of complex networks: networks having this property have an average path length similar to random networks, but a much higher clustering coefficient. Later, Latora and colleagues \cite{Latora2001} reformulated the small-world property in terms of global and local efficiency. \medskip

Barab\'asi and colleagues introduced the BA scale-free model of complex networks, where the mechanisms of growth and preferential attachment lead to a degree distribution following a power law \cite{BarabasiA.-L;Albert1999}. Arguably, the behavior of the world-wide-web can be explained with this model \cite{Barabasi2000}. The BA model has a low clustering coefficient, contrarily to random networks, then in \cite{Klemm2002} is defined a KE generative model of highly-clustered scale-free networks. In Refs. \cite{Holme2002d} and \cite{Ravasz2003} can be found other network models with power-law degree distribution, short average path length and high clustering coefficient.\medskip

Empirical research identified three structural classes of small-world networks: scale-free networks, with a degree distribution with a tail that decays as a power law, broad-scale or truncated scale-free, with a power-law regime followed by a  sharper cutoff and single-scale networks, with a  fast-decaying tail \cite{Amaral2000}. The emergence of the first two networks can be explained by optimization mechanisms, complementary to growth and preferential attachment \cite{Cancho2003}.

\section{Airport networks}

The first analysis of the world airport network (WAN) were reported by \cite{Guimera2005, Guimera2004} as unweighted network and by \cite{Barrat2004} as weighted network.\medskip

The first analysis of the Chinese airport network were carried out by \cite{Li2004b}. Later, \cite{Lin2012a} makes a complex network and spatial analysis of the Chinese air transport network. Other regional studies of airport networks are:

\begin{itemize}
	\item India: \cite{Bagler2008}
	\item Brazil: \cite{DaRocha2009}
\end{itemize}

\cite{Zanin2013} is a review of studies of air transport networks. The section of dynamic models of air transport is of special interest, especially for the referenced studies of connectivity \cite{Burghouwt2005} and \cite{Malighetti2008}. 

\section{Air route networks}

Analysis of air route networks: \cite{Cai2012a, Sun2014, Sun2015, Du2017b}.

\section{Static robustness}

Introduction of the problem of static robustness: \cite{Albert2000}. \medskip

Analysis of static robustness in terms of network efficiency: \cite{Crucitti2003}  for BA and KE graphs, \cite{Crucitti2004b} for BA graphs. Scale-free tend to be robust to errors, but sensitive to attacks. In Ref. \cite{Holme2002a} is performed an analysis of the effect of edge and node removal for several real-world and models of complex networks. \medskip

In \cite{Paul2005} is obtained the network that optimizes robustness for the sum of critical threshold for random deletion \cite{Cohen2000a} and for deletion of central nodes \cite{Cohen2001a}. \medskip

Measures of network robustness: algebraic connectivity \cite{Fiedler1973}, effective graph resistance \cite{Ellens2011} and mutiscale vulnerability \cite{Boccaletti2007}.\medskip

Schneider and colleagues \cite{Schneider2011} define a unique network robustness measure (in \cite{Hong2017} is described a procedure to compute an approximate value). They optimize this measure rewiring network nodes, maintaining the degree of each node. Refinements of this strategy can be found in \cite{Louzada2013, Yang2015, Zhou2014}.


\section{Cascading failures}

In Ref. \cite{Motter2002} is defined a global load-based model of cascading failures. Node load is equal to number of shortest paths passing through it. Nodes with load above capacity are removed from the network. Evolution is tracked through size of connected component. In Ref. \cite{Crucitti2004} is defined an similar model, and track evolution of damage through network efficiency. \medskip

In Ref. \cite{Wang2008} is defined a local weighted flow redistribution rule model for cascading failures. In this model, the load of congested edges is redistributed locally, among the adjacent edges.\medskip

In Ref. \cite{Moreno2002} is defined a fiber-bundle model of cascading failures, and in \cite{Huang2006} a sandpile model for a geographical network.\medskip

Holme and colleagues analyze the cascading failures for node \cite{Holme2002b} and edge \cite{Holme2002c} overload in growing networks. As the network grows, values of node or edge betweenness increase, so when they reach a critical point a cascading failure may arise.\medskip

\medskip

\nocite{*}

\bibliographystyle{custom2}
\bibliography{bib_networks}
\end{document}