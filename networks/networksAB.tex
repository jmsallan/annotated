\documentclass[12pt]{article}
\usepackage{geometry}   
\geometry{a4paper}
\usepackage[english]{babel}
\usepackage[utf8]{inputenc}
\title{Annotated bibliography on complex networks}
\author{Jose M Sallan}
\date{}
\begin{document}

\maketitle

\section{Network models}

The first attempt to model real-complex networks was the Erdös-R\'enyi model of random networks. As an explanation of real-world network being more clustered than random networks, Watts and Strogatz \cite{Watts1998a} defined and modeled the small-world property of complex networks. Barab\'asi and colleagues introduced the scale-free model of complex networks, where the mechanisms of growth and preferential attachment lead to a degree distribution following a power law \cite{BarabasiA.-L;Albert1999}. Arguably, the behavior of the world-wide-web can be explained with this model \cite{Barabasi2000}. Later empirical research identified three structural classes of small-world networks: scale-free networks, with a degree distribution with a tail that decays as a power law, broad-scale or truncated scale-free, with a power-law regime followed by a  sharper cutoff and single-scale networks, with a  fast-decaying tail \cite{Amaral2000}. The emergence of the first two networks can be explained by optimization mechanisms, complementary to growth and preferential attachment \cite{Cancho2003}.

\section{Static robustness}

In \cite{Paul2005} is obtained the network that optimizes robustness for the sum of critical threshold for random deletion \cite{Cohen2000a} and for deletion of central nodes \cite{Cohen2001a}.

\nocite{*}

\bibliographystyle{plain-annote}
\bibliography{bib_networks}
\end{document}